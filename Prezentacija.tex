\documentclass{beamer}
\usepackage[utf8]{inputenc} 
\usepackage[croatian]{babel}
\usepackage{graphicx}

\title{Github vs Bitbucket}
\author{Luka Otović, Ivo Santini, Mauro Copetti}
\date{12.siječnja 2018.}
\institute{Tehnički fakultet Rijeka}


\begin{document}
	
	\frame{\titlepage}
	
	\begin{frame}                                         
		\frametitle{GitHub}

		%općenito o GitHubu


			
	\begin{figure}[h!]
		\begin{center}
			\includegraphics[scale=0.10]{macka.png}
			\caption{Maskota GitHub-a}
		\end{center}
	\end{figure}

	\end{frame}                  %frame 1




	\begin{frame}              
		\frametitle{Bitbucket}


		%općenito o bitbucketu

		\begin{figure}[h!]
			\begin{center}
				\includegraphics[scale=0.10]{macka.png}
				\caption{Maskota GitHub-a}
			\end{center}
	\end{figure}

	\end{frame}                %frame 2








	\begin{frame}        

	slika GitHub vs Bitbucket + naslov     


	\end{frame}                             %frame 3


	\begin{frame}
		sliak velika za naslov `` SLIČNOSTI''

	\end{frame}


	\begin{frame}
		\frametitle{Dizajn}
		
		%dodati title Korištenej GitHub-a i Bitbucket-a
		\begin{itemize}
			\item Git servisi za kontrolu verzija, odnosno omogućuju uvid u izmjenu programskog koda.
			\item Koriste ih uglavnom programeri.


		\end{itemize}

	\end{frame}                              %frame 4





	\begin{frame}
		\frametitle{Dizajn} %ne radi
		dkmfsdkf
		Slika GitHub-a kako izgleda

	\end{frame}

	\begin{frame}
		\frametitle{Dizajn}
		da,flalalooka
		Slika Bitbucket-a kako izgleda


	\end{frame}                           %frame 4 i 5 





	\begin{frame}


	\end{frame}

	\begin{frame}


	\end{frame}

	\begin{frame}


	\end{frame}

	\begin{frame}


	\end{frame}

	\begin{frame}


	\end{frame}

	\begin{frame}


	\end{frame}




\end{document}

