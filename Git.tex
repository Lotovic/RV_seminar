\documentclass{article}

\usepackage[croatian]{babel}
\usepackage[utf8]{inputenc}
\usepackage{hyperref}
\usepackage{graphicx}

\title{GitHub}
\date{9.1.2018}
\author{Luka Otović, Ivo Santini, Mauro Copetti}


\begin{document}

	\maketitle
	\newpage
	\tableofcontents
	\newpage

	GitHub je web-baziran Git servis za kontrolu verzija, odnosno omogućuje uvid u izmjene programskog koda. Uglavnom ga koriste programeri u različite svrhe npr. za veće projekte,
	lakše debuganje, lakše upravljanje zadacima i poboljšvanje kodova. Od travnja 2017., GitHub postaje najvećim domaćinom izvornog koda u svijetu premašujući
	20 miljuna korisnika i 57 miljuna spremišta.
	\\
	Projekti na GitHubu mogu se pristupiti i manipulirati korištenjem standardnog sučelja Git-a iz naredbenog retka, a sve standardne
	Git naredbe podržane su od strane GitHub-a.

	\section{\textbf {Razvoj GitHuba}}
		Razvoj je započeo 19. listopada 2007. Developeri GitHub-a Tom Preston-Werner, Chris Wanstrath i PJ Hyett su ga objavili
		u travnju 2008. godine, a nekoliko mjeseci prije je bila puštena beta verzija.\\
		Puno različitih firma se pridružilo GitHubu npr:
		\begin{itemize}
			\item 2008. godine - Reddit
			\item 2008. godine - Yahoo
			\item 2009. godine - Twitter
			\item 2009. godine - Facebook
			\item 2010. godine - Pinterest
			\item 2012. godine - Google
		\end{itemize}
	
		Jedini kompetetivan GitHub-u je Bitbucket koji je pokrenut (objavljen) 2008. godine.
		\\
		Puna vremenska crta razvoja GitHuba nalazi se na linku:\\
		\url{https://en.wikipedia.org/wiki/Timeline_of_GitHub}
		\\
		\\
	\section{\textbf {Prednosti GitHub-a}:}
		\begin{enumerate}
			\item Nudi sve naredbe za upravljanjem kontrolom verzija putem standardnog Git sučelja, kao i dodavanjem vlastitih značajki.
			\item Privatni ili otvoreni (javni) repozitoriji (spremišta) za održavanje softverskih projekata
			\item Omogućuje registriranim i neregistriranim korisnicima pregledavanje javnih spremišta na web mjestu.	
			\newpage
			\item Pruža funkcije slične društvenim mrežama:
			\begin{itemize}
				\item novosti
				\item grafikon društvene mreže (kako bi se prikazalo kako programeri rade na njihovim verzijama i razvijaju program)

			\end{itemize}
			\item Bilo tko može pregledavati i preuzimati sa javnog spremišta
			\item Stvara povijest našeg razvijanja koda
			\item ...
		\end{enumerate}
		Kako bi korisnik doprinio sadražj web sučelju, napravio svoji repozitoriji, radio na svojem projektu,.. treba
		prvo otvoriti svoji račun na \href{https://github.com/github}{GitHub-u}.\\
		Softver koji pokreće GitHub je napisan od strane Github,Inc. koristeći Ruby on Rails i Erlang.
		\\
		\\
		\textbf{GitHub podržava jako puno formata i značajki poput}:
		\begin{itemize}
			\item Markdown-a, LaTeX i njima sličnima
			\item Wikis
			\item Grafikoni
			\item Emojis
			\item Email obavijesti
			\item ...\\
		\end{itemize}

		Svaki korisnik ili programer koji namjerava koristiti GitHub treba pročitati softversku licencu kako bi se utvrdilo ispunjava li
		njegove potrebe. Uvjeti pružanja usluge navode:\textbf{''Postavljanjem vaših spremišta da budu pregledani javno, pristajete da dopustite drugima da pregledavaju i stvaraju nove grane
		na vašem projektu''}
		\\
		\\
		GitHub je razvio nove online usluge koje služe za točno određene situacije npr:
		\begin{enumerate}
			\item GitHub Enterprise - dizajniran za upotrebu velikih timova za razvoj poslovnog softvera
			\item GIST - se upotrebljava za više manjih projekata; služi za hosting kodnih isječaka
		\end{enumerate}

	\section{\textbf {Obrazovni program}}
		GitHub je pokrenuo novi program nazvan ''GitHub Student Developer Pack'' kako bi studentima omogućio besplatan pristup popularnim razvojnim alatima i uslugama.
		U program su uključene razne firme medju kojima su: Bitnami, Crowdflower, DigitalOcean, DNSimple, Unreal Engine,.. kako bi ostvarili svoji obrazovni program.
		\\

	\section{\textbf {Cenzura}}
	\begin{itemize}
		\item 3.prosinca 2014. bio je blokiran u Rusiji na nekoliko dana zbog priručnika za samoubojstvo koje su objavili korisnici
		\item 31.prosinca 2014. blokiran je u Indiji radi informacija o ''ISIS'', 10.siječnja 2015. GitHub je bio odblokiran
		\item 26.ožujka 2015. GitHub je bio DDoS-an i nije radio sljedečih 118 sati
		\item 8.listopada 2016. bio je blokiran u Turskoj kako bi zaustavili javno objavljivanje e-mailova sa hakiranog računa koji pripada ministru energetike\\
	\end{itemize}

	GitHub isto ima svoju maskotu:
	\begin{figure}[h!]
		\begin{center}
			\includegraphics[scale=0.15]{macka.png}
			\caption{Maskota GitHub-a}
		\end{center}
	\end{figure}

\end{document}